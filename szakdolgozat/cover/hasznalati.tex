\pagestyle{empty}

\noindent \textbf{\Large Adathordozó használati útmutató}

\vskip 1cm


A mellékelt adathordozón a következő jegyzékek és fájlok találhatók

\begin{itemize}
\item Program mappa, melyen belül található egy Frontend és egy Backend mappa. Ezeken belül találhatóak a forráskódok,
\item Szakdolgozat mappa, ahol a Szakdolgozat LaTeX kódja található,
\item Szakdolgozat.pdf, mely a szakdolgozatot tartalmazza, PDF formátumban,
\end{itemize}

Fronted futtatásához a \textit{Visual Studio Code} fejlesztőkörnyezet szükséges. A VSCode tartalmaz egy terminált, ott kell lefuttatni az \textit{npm install} parancsot, ekkor létrejön egy \textit{node\_modules} mappa amibe letöltődik az összes szükséges fájl. Következő lépésben az \textit{ng serve} paranccsal el is indíthatjuk a programot ami a \textit{localhost:4200}-on fog futni.

Backend futtatásához az \textit{IntelliJ IDEA Community Edition}-re lesz szükségünk ami ingyenes, mindenki számára elérhető. Miután ott megnyitottuk a mappát akkor futtathatjuk is a programot ami a {localhost:8080}-on fog futni.