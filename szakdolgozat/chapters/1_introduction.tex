\Chapter{Bevezetés}

A mai világban kevés olyan ember van, aki ne használna személygépjárművet, és ennek köszönhetően nagyon gyorsan cserélődnek az autók a használt autó piacon. Nem sok mindenki engedheti meg magának a legújabb autóknak a megvételét, főleg akkor ha fiatal, és még nincs teljes munkaidős állása. Erre nagyon sok használt autó hirdető webalkalmazás épül, ahol nagyon sok ember keresi nap mint nap a magának megfelelő autót, és mivel gyorsan cserélődnek a használt autók, előbb-utóbb meg is fogja találni a számára megfelelőt.

Emellett nagyon sok ember jobban preferálja az online vásárlást, és így rengeteg webshop elérhető az Interneten. Léteznek olyan weboldalak is, melyek célja az aggregá-
lás, ami azt jelenti, hogy itt nagyon sok webáruház termékei megtalálhatóak, és ez így nagyban gyorsítja a vásárlásnak a folyamatát, mivel nem kell több oldalt is átnéznie, hanem egy helyen megtalál mindent.

Szakdolgozatomban azt szeretném bemutatni, hogy milyen használt autó és aggre-
gáló weboldalt ismerek. Betekintést fogunk kapni egy használt autó aggregáló webalkal-
mazás fejlesztésébe a tervezéstől a megvalósításig. 

A szakdolgozat elkészítéséhez majd fel tudom használni a Webtechnológiák nevű tantárgyból és a szakmai gyakorlatom alatt szerzett tudásomat. Ez alapján esett a választásom a Java (Spring Boot), Angular és PostgreSQL technológiák használatára.

A dolgozatban bemutatásra kerül a megtervezett webalkalmazás kinézete, felépítése milyen adattáblákkal fog dolgozni, milyen végpontokat fog meghívni. Ezután ezt a megtervezett webalkalmazást fogom részletezni és bemutatom a fejlesztés folyamatát.  A dolgozat végén azt fogom kifejteni, hogy milyen jövőbeli tervek lesznek az webalkalma-
zással kapcsolatban.