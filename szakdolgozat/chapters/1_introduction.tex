\Chapter{Bevezetés}

A mai világban kevés olyan ember van aki ne használna személygépjárművet és ennek köszönhetően nagyon gyorsan cserélődnek az autók a használt piacon. Nem sok mindenki engedheti meg magának a legújabb autóknak a megvételét, főleg akkor ha fiatal és még nincs teljes munkaidős állása. Erre nagyon sok használt-autó hírdető webalkalmazás épül, ahol nagyon sok ember keresi nap mint nap a magának megfelelő autót és mivel gyorsan cserélődnek a használt-autók előbb utóbb meg is fogja találni a számára megfelelőt.

Emellett nagyon sok ember jobban preferálja az online vásárlást, és így rengeteg webshop elérhető az Interneten. Léteznek olyan weboldalak is, melyek célja az aggregálás ami azt jelenti, hogy itt nagyon sok webáruház termékei megtalálhatóak, és ez így nagyban gyorsítja a vásárlásnak a folyamatát, mivel nem kell több oldalt is átnéznie hanem egy helyen megtalál mindent.

Szakdolgozatomban azt szeretném bemutatni, hogy milyen használt-autó és aggregáló weboldalt ismerek. Betekintést fogunk kapni egy használt-autó aggregáló webalkalmazás fejlesztésébe a tervezéstől a megvalósításig. 

A szakdolgozat elkészítéséhez majd feltudom használni a Webtechnológiák nevű tan-
tárgyból és a szakmai gyakorlatom alatt szerzett tudásomat. Ez alapján esett a választásom a Java(Spring Boot), Angular és PostgreSQL technológiák használatára.