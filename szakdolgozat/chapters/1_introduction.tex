\Chapter{Bevezetés}

A mai világban kevés olyan ember van aki ne használna személygépjárművet és ennek köszönhetően nagyon gyorsan cserélődnek az autók a használt piacon. Mivel nem sok mindenki engedheti meg magának a legújabb autóknak a megvételét, főleg akkor ha fiatal és még nincs teljes munkaidős állása. Erre nagyon sok használt-autó hirdető webalkalmazás épül, ahol nagyon sok ember keresi nap mint nap a magának megfelelő autót és mint már említettem mivel gyorsan cserélődnek a használt-autók előbb utóbb meg is fogja találni a számára megfelelőt.

Emellett nagyon sok ember jobban preferálja az online vásárlást és így rengeteg webshop elérhető az interneten.Léteznek olyan weboldalak is ,melyek aggregáló funkció-
júak ami azt jelenti, hogy itt nagyon sok webáruház terméke megtalálhatóak és ez így nagyban gyorsítja a vásárlásnak a folyamatát, mivel nem kell több oldalt is átnéznie hanem, egy helyen megtalál mindent.

Szakdolgozatomban azt szeretném bemutatni, hogy milyen használt-autó és aggre-
gáló weboldalt ismerek. Betekintést fogunk látni egy használt-autó aggregáló webalkal-
mazás fejlesztésébe a Tervezéstől a megvalósításik. 

Szakdolgozat elkészítéséhez majd feltudom használni a Webtechnológiák nevű tan-
tárgyból és a szakmai tapasztalatom alatt szerzett tudásomat és ez alapján esett választás a Java(Spring Boot), Angular és PostgreSQL tecnológiák használatára.

Úgy szeretném elkészíteni a szakdolgozatomat, hogy az a valós tudásomat tükrözze amit az egyetemi éveim alatt gyűjtöttem össze.