\Chapter{Összefoglalás}

A szakdolgozatom egy Használt-autó aggregáló webalkalmazás tervezése és elkészítése volt amihez tartozik az ehhez használt technológiák és eszközök bemutatása.

A szakdolgozatom első elméleti részében megismertettem azokat a technológiákat és eszközöket amiket használtam én magam a fejlesztés közben. A szakdolgozatom elkészítése alatt sokkal nagyobb betekintés nyertem Webfejlesztésbe és azon belül job-ban megismerhettem a Spring Boot és Angular keretrendszert és a PostgreSQL adat-
bázist.

Későbbi részben betekintést nyerhettünk a hasonló célú szoftverekre és szolgáltatá-sokra. Megtudtuk azt hogy kettő oldal között milyen nagy különbségek lehetnek, hiába, hogy ugyan azt a célt szolgálja.

Megismerkedhettünk azzal hogy egy alkalmazás fejlesztése előtt milyen tervezési lépésre vannak szükségek és hogy mennyire is fontosak, hogy utána zökkenő mentesen menjen a fejlesztés és ne később derüljenek ki egyes dolgok, hogy mik is kellenek még az alkalmazásba

Legvégül betekintést látunk a megvalósításra, hogy a mappa szerkezet hogyan néz ki, mennyi külön féle feladatot hajt végre egy alkalmazás miközben fut és ezt egy átlag felhasználó nem veszi észre mert számára a felhasználói felület ezt elrejti és nagyon könnyedén tudja használni az általunk elkészített alkalmazást.

Személy szerint én úgy gondolom, hogy egy gépjármű aggregáló webalkalmazás nagyon fontos és jó dolog az emberek számára, mivel mikor bárki autót szeretne vásárolni akkor rengeteg oldalon szétnéz és ezzel rengeteg időt vesz el magától mire megtalálja a megfelelő autót. Ha van egy ilyen aggregáló weboldal akkor sokkal köny-nyebben és gyorsabban megtalálja a neki megfelelő autót, mivel az összes egy helyen elérhető számára. 
