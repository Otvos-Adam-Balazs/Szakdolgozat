\Chapter{Összefoglalás}

A szakdolgozatom egy használt-autó aggregáló webalkalmazás tervezése és elkészítése volt, amihez tartozik az ehhez használt technológiák és eszközök bemutatása.

A szakdolgozatom első elméleti részében ismertettem azokat a technológiákat és eszközöket, amiket használtam a fejlesztés közben. A szakdolgozatom elkészítése alatt sokkal nagyobb betekintés nyertem a webfejlesztésbe, és azon belül jobban megismerhettem a Spring Boot és Angular keretrendszert és a PostgreSQL adatbázist.

A későbbi részben betekintést nyerhettünk a hasonló célú szoftverekbe és szolgáltatá-sokba. Megtudtuk azt, hogy két oldal között milyen nagy különbségek lehetnek, hiába, hogy ugyan azt a célt szolgálják.

Megismerkedhettünk azzal, hogy egy alkalmazás fejlesztése előtt milyen tervezési lépésre van szükség, és hogy mennyire fontos, hogy zökkenő mentesen menjen a fejlesztés, és ne később derüljenek ki, hogy mik is kellenek még az alkalmazásba

Legvégül betekintést kapunk a megvalósításra, hogy a mappa szerkezet hogyan néz ki, mennyi különféle feladatot hajt végre egy alkalmazás miközben fut, és ezt egy átlag felhasználó nem veszi észre, mert számára a felhasználói felület ezt elrejti, és nagyon könnyedén tudja használni az általunk elkészített alkalmazást.

Személy szerint úgy gondolom, hogy egy gépjármű aggregáló webalkalmazás nagyon fontos és jó dolog az emberek számára, mivel amikor valaki autót szeretne vásárolni, akkor rengeteg oldalon szétnéz, és ezzel rengeteg időt vesz el magától mire megtalálja a megfelelőt. Ha van egy ilyen aggregáló weboldal akkor sokkal könnyebben és gyorsabban megtalálja a neki megfelelő autót, mivel az összes egy helyen elérhető számára. 
