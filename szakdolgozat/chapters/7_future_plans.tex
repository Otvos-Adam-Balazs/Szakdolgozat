\Chapter{Jövőbeli tervek}

\Section{Több oldal bevonása}

A Jövőben lehetne több oldalt is bevonni az oldalhoz, hogy a felhasználók számára még könnyebben sokkal több autó álljon rendelkezésükre a keresésben, és így egyszerűbben megtalálhatnák a nekik megfelelő autót. Akár külföldi oldalak bevonása is szóba jöhet, hogy ne csak a magyarországi autókat találják meg.

\Section{Összehasonlítás}

Lehetőség lenne az autók összehasonlítására különböző paraméterek szerint mint év-
járat, üzemanyag, km, és így könnyebben ki lehetne választani a szimpatikus autók közül a legmegfelelőbbet.

\Section{További statisztika}

További statisztikai vizsgálatok elkészítése is célszerű lenne, hogy a felhasználók jobban lássák azt, hogy milyen időszakokban milyen autókat választanak jobban a vásárlók. Ilyen vizsgálatok lennének például a következők:

\begin{itemize}
\item Üzemanyag típus, itt megtudnánk azt nézni hogy az év melyik szakaszában milyen üzemanyag fogyasztással rendelkező autók a kelendőbbek.

\item Évjárat, itt azt tudnánk nézni, hogy átlagban milyen évjáratú autókat keresnek a felhasználók, továbba, hogy inkább fiatal vagy öregebb autót vásárolnak.

\item Km-re váló statisztika, ebből arra kapnánk következtetést, hogy mennyire számít a felhasználók számára, hogy mennyi kilométer szerepel a keresett autókban, és azt is megtudnánk, hogy átlagban mennyi kilométerrel rendelkező autót vásárolnak a felhasználók.
\end{itemize}

\Section{Más típusú gépjárművek}

Későbbiekben akár ki lehetne bővíteni az oldalt olyan irányba is, hogy ne csak személygépjárművek aggregálására legyen lehetőség, hanem motorok, haszongépjárművek, csónakok, hajók, vagy akár autóbuszok aggregálására is alkalmas legyen. Ezzel még több felhasználó tudná hasznát vennis.