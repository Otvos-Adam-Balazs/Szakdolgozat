\Chapter{Backend Dokumentáció}

Ebben a fejezetben azt fogom bemutatni, hogy egy Java (Spring Boot) applikáció hogyan épül fel.
\Section{Projekt inicializálás}

A projektet a Spring Boot Initializr segítségével hoztam létre.

\begin{figure}[h]
\centering
\includegraphics[scale=0.37]{images/Spring_init.png}
\caption{Spring Boot Initializr \cite{SpringInit}}
\label{fig:Spring_Boot_Initializr}
\end{figure}

Ahogy a \ref{fig:Spring_Boot_Initializr}. ábrán is látszik, ki lehet választani, hogy milyen típusú projekt legyen, milyen nyelven szeretnénk elkészíteni, a Spring Boot verzióját, metaadatokat, milyen Java verzióra szeretnénk fejleszteni.

Függőségeket tudunk megadni. A létrehozáskor a Spring Web-et, a Spring Data JPA-t és a Lombok-ot adtam hozzá.

A függőségek leírása a \textit{pom.xml} fájlban találhatóak meg. 

Mikor mindent kiválasztottunk a "GENERATE" gombra kattitnva le is generálja nekünk az alap projektet.
\newpage

Az elkészített projekt mappaszerkezete a \ref{fig:Intelij-mappaszerkezet}. ábrán látható.
\begin{figure}[h]
\centering
\includegraphics[scale=0.7]{images/Intelij-mappaszerkezet.png}
\caption{Intelij-mappaszerkezet}
\label{fig:Intelij-mappaszerkezet}
\end{figure}

Maga a szerver a \textit{http://localhost:8080/} címen fut.

\Section{Adatbázis létrehozása} 

Első lépésben az Adatbázis létrehozásához 2 függőséget kellett használnom amit a \textit{pom.xml} fájlba kellett elhelyezni:
\begin{java}
<dependency>
	<groupId>org.springframework.boot</groupId>
	<artifactId>spring-boot-starter-data-jpa</artifactId>
</dependency>

<dependency>
	<groupId>org.postgresql</groupId>
	<artifactId>postgresql</artifactId>
	<scope>runtime</scope>
</dependency>
\end{java}

Miután a behelyezés megtörtént, az adatbázisunkat hozzá kell kapcsolni a szerve-
rünkhöz, ami az application.properties fájlban történik. Én PostgreSQL-t adatbázist használok az adatok tárolására és a kapcsolat így jön létre a DB és a szerver között:

\begin{java}
spring.datasource.url=jdbc:postgresql://localhost:5432/cardb
spring.datasource.username=postgres
spring.datasource.password=Password
spring.jpa.properties.hibernate.dialect=
  org.hibernate.dialect.PostgreSQLDialect
spring.jpa.properties.hibernate.format_sql=true
\end{java}

Az URL az adatbázis szerver és maga az adatbázis eléréséhez szükséges. A szerver a 5432 porton fut és a cardb nevű adatbázisra lesz szükségünk.
A \textit{username} és \textit{password} ahhoz kell, hogy tudjuk magunkat hitelesíteni.

Következő lépésben létrehoztam a 3 adatbázis táblát (\textit{car}, \textit{search\_data}, \textit{users})ami-
hez létrehoztam mindegyikhez egy \textit{entity}-t és egy \textit{repository}-t.

\textit{Entity}-n belül tudjuk megadni, hogy a táblán belül milyen oszlopok legyenek, és hogy mi legyen az elsődleges kulcs.

Példa egy \textit{Entity} Java osztályra, ami a users nevű táblát hozza létre:

\begin{java}
@Getter
@Setter
@NoArgsConstructor
@AllArgsConstructor
@Entity
@Table(name="users")
public class User {
    @Id
    @GeneratedValue(strategy = GenerationType.IDENTITY)
    @Column(name = "user_id")
    private int id;

    @Column(name = "login")
    private String login;

    @Column(name = "password")
    private String password;

    @Column(name = "authority")
    private String authority;
}
\end{java}

A \textit{Repository}-ra azért van szükségünk hogy a szerverünk tudjon kommunikálni az adatbázissal, tehát hogy tudjon adatot lekérdezni, felvinni törölni vagy éppen módosíta-
ni, és ehhez ki kell terjeszteni a \textit{JpaRepository} osztályt, ami tartalmaz alapvető művele-
teket.

Példa egy \textit{Repository} osztályra:

\begin{java}
@Repository
public interface UserRepository
extends JpaRepository<User, Integer> {
    List<User> deleteByLogin(String login);
    User findUserByLogin(String login);
}
\end{java}
\newpage

\Section{Autók importálása az adatbázisba}

Az adatok megszerzéséhez WebCrawlert \cite{WebCrawler} használok, ami azt jelenti, hogy az adato-
kat a weboldalak HTML-tartalmából gyűjtöm össze. Ehhez az egész weboldal felépítését át kell hozzá néznem, mivel tag-ek vagy éppen \textit{cssQuery}-vel lehet egyes adatokat kinyerni. 

Esetemben ez azért volt elég nehezen megoldható, mert arra is figyelni kellett, hogy vannak olyan autók a honlapon, amik nem rendelkeznek az összes adattal, ezért ki kellett szűrni az úgymond hibásan feltöltötteket. Olyan adatok hiányoztak sokszor, mint például: lóerő, hengerűrtartalom vagy éppen a futott kilométer nem volt megadva. Nagyjából az autóknak az 5\%-a volt hibás.

A lekérdezett adatok mennyisége attól függ, hogy hány oldalt fésülök át a Crawler segítségével, és hogy mennyi olyan autó áll ezeken az oldalakon rendelkezésre, ami az összes adattal rendelkezik.

Minden egyes adatkinyerés lefuttatása folyamán 600-900 darabszám közötti gépjár-
művet sikerült összegyűjteni. 

Lent látható egy példa egy WebCrawler-re.

\begin{java}
public void saveJoAutok(String joAutok)
  throws IOException, InterruptedException {
    int i = 30;
    try {
      Document doc = Jsoup.connect(joAutok +"?page=" +i)
                          .get();

      Elements cars = doc.select("a.item");
      for (Element carD : cars) {
      Car car = new Car();

      String price = carD.select("div.price").text()
                         .replaceAll(" ", "");
      double priceNumber = Integer
    	  .parseInt(price.substring(0, price.indexOf("F")));
      car.setPrice(priceNumber);


      car.setImage(carD.getElementsByTag("img")
      .attr("abs:data-src"));
      car.setLink(carD.select("a").attr("abs:href"));

      String CarName[] = carD.select("div.h2-top")
      .text().split(" ");
      String CarData[] = carD.select("span.dotted")
      .text().split(" ");

      car.setMakes(CarName[0]);
      car.setModel(CarName[1]);
      System.out.println(CarData);
      car.setFuelType(CarData[7]);
      String Age[] = carD.select("b").text().split(" ");

      if (car.getFuelType().equals("Benzin") ||
        car.getFuelType().equals("Dizel")) {
      try {
        car.setAge(Integer.parseInt
        (Age[2].substring(0,Age[2].indexOf("."))));
        car.setEngine(Integer.parseInt(CarData[5]));
        car.setHp(Integer.parseInt(CarData[3]));
        car.setMileage(CarData[0] + CarData[1]);
        carRepository.save(car);
       }catch (Exception e) {
             System.out.println("Some Data is Bad!");
       }
      } else {}
        }
      }catch (Exception e) {
        System.out.println("Bad Price");
      }
   }
}
\end{java}

\Section{Autók lekérdezése az adatbázisból}

Ha valaki autókra szeretne keresni akkor a \textit{/get} végpont hívódik, meg ami Requestpara-
métereket vár. Rengeteg paraméter kell egy ilyen lekérdezéshez, amit opcionálisan ki is lehet hagyni, így arra volt szükségem hogy dinamikus lekérdezést tudjak létrehozni, amihez a Criteria API-t használom.

JPA Specification teszi azt lehetővé, hogy dinamikus legyen a lekérdezés ami a Criteria API-ra épül, tehát nem kell minden adatnak szerepelnie benne, vagy éppen úgy is működik, ha nincsen benne szereplő kritérium, hogy mi alapján kérdezze le az adatokat. (Akkor az összeset lekérdezi az adatbázisból.) A Predicate interfészével tudjuk összekötni a Query-nket a vizsgálat után.

\begin{java}
public List<Car> getCars(String makes, String model,
      String fuelType, String minAge,String maxAge,
      String minEngine, String maxEngine, 
      String minPrice,String maxPrice) {
  return carRepository.findAll(new Specification<Car>() {
    @Override
    public Predicate toPredicate(Root<Car> root,
     CriteriaQuery<?> cq, CriteriaBuilder cb) {
        Predicate p = cb.conjunction();
        if (!StringUtils.isEmpty(makes)) {
            p = cb.and(p, cb.equal(root.get("makes"), makes));
        }
        if (!StringUtils.isEmpty(model)) {
            p = cb.and(p, cb.equal(root.get("model"), model));
        }
        if (!StringUtils.isEmpty(fuelType)) {
            p = cb.and(p, cb.equal(root.get("fuelType")
                                       , fuelType));
        }
        if (!StringUtils.isEmpty(minAge) && 
            !StringUtils.isEmpty(maxAge) &&
            Integer
            .parseInt(minAge)<Integer.parseInt(maxAge)) {
            p = cb.and(p, cb.between(root.get("age"),
            Integer.parseInt(minAge),Integer
                                    .parseInt(maxAge)));
        }
        if (!StringUtils.isEmpty(minEngine) &&
            !StringUtils.isEmpty(maxEngine) &&
            Integer.parseInt(minEngine)<Integer
                                       .parseInt(maxEngine)) {
            p = cb.and(p, cb.between(root.get("engine"),
            Integer.parseInt(minEngine),Integer
                                       .parseInt(maxEngine)));
        }
        if (!StringUtils.isEmpty(minPrice) &&
            !StringUtils.isEmpty(maxPrice) &&
            Integer.parseInt(minPrice)<Integer
                                       .parseInt(maxPrice)) {
             p = cb.and(p, cb.between(root.get("price"),
             Integer.parseInt(minPrice),Integer
                                       .parseInt(maxPrice)));
         }
         return p;
     }
  });
}
\end{java}

Ahogy a kódban is látható, úgy oldottam meg azoknak a paramétereknek a vizsgála-
tát, ott ahol -tól -ig értékek vannak, hogy amikor a minimum érték nagyobb mint az maximum, akkor nem kerül bele a lekérdezésbe, és így nem fogunk hibát kapni sosem a keresésnél.

\Section{Bejelentkezés és Regisztrációs}
A regisztráció és bejelentkezés elkészítéséhez a Spring Security keretrendszert \cite{SpringSecurity} hasz-
náltam, mert nagyban megkönnyíti a fejlesztési folyamatot, mivel biztosít előre elkészí-
tett modulokat.

\subsection{Regisztráció}
Ahhoz, hogy az oldalt használni lehessen  a regisztráció az egyik legfontosabb lépés. Ez felhasználónév és  jelszó megadásával lehetséges. 

Ha valaki be akar regisztrálni, akkor a  \textit{/register} végpont hívódik meg és a body-ban várja a felhasználónév és jelszó párosítást. Ennek a vizsgálatra az alábbi kódrészlet a felelős.

\begin{java}
public String register(UserDto userDto) {
    User user;
    if(ObjectUtils.isEmpty(userDto.getLogin())) {
      return "2";
    } else {
        user = userRepository
          .findUserByLogin(userDto.getLogin());
    }
    if (ObjectUtils.isEmpty(userDto.getPassword())) {
      return "2";
    } else {
        if (user == null) {
           User newUser = new User();
           newUser.setLogin(userDto.getLogin());
           newUser.setPassword(bCryptPasswordEncoder
                      .encode(userDto.getPassword()));
           newUser.setAuthority("USER");
           userRepository.save(newUser); 
           return "0";
        } else {
           return "1";
        }
    }
}
\end{java}

Először megvizsgálja, hogy mind a kettő adat rendelkezésre áll-e a bejelentkezéshez, és ha nem, akkor 2 értéket ad vissza. Ha minden rendben van megnézi, hogy az adat-
bázisban szerepel-e már ilyen nevű felhasználó, ha nem akkor létrehozza.


A jelszó be lesz Hash-elve, amihez a Spring Security  keretrendszernek a BCrypt kódolási mechanizmusát használtam, ami egy 60 karakter hosszú Stringet generál.

Egy új felhasználó alapértelmezett jogosultsága egy USER lesz, amit később egy másik Admin megváltoztathat.

\subsection{Bejelentkezés}

Regisztráció után jöhet a bejelentkezés. A megfelelő felhasználónév és jelszó kombi-
nációjával tudunk authentikálni. Az alábbi kódrészlet azt mutatja meg, hogy hogyan állítottam be azt, hogy mely oldalak azok, amelyek authentikáció nélkül is elérhetőek. Ez a 2 oldal a bejelentkezés és a regisztráció.

\begin{java}
    @Override
    protected void configure(HttpSecurity http)
     throws Exception {
        http.cors();
        http.authorizeRequests()
        .antMatchers(HttpMethod.POST,"register")
        .permitAll();
        http.csrf().disable()
          .authorizeRequests()
          .antMatchers("authenticate")
          .permitAll()
          .antMatchers(HttpHeaders.ALLOW)
          .permitAll()
          .anyRequest().authenticated()
          .and()
          .exceptionHandling()
          .authenticationEntryPoint(
           jwtAutheticationConfig)
          .and()
          .sessionManagement()
          .sessionCreationPolicy(
          SessionCreationPolicy.STATELESS);
          http.addFilterBefore(
          jwtRequestFilter,
          UsernamePasswordAuthenticationFilter.class);
}
\end{java}

Bejelentkezés után érhetjük el azokat a végpontokat, de csak azokat, amihez van jogosultságunk. Ez azért nagyon fontos, hogy ne tudjon egy sima felhasználó olyan dolgokat megtenni mint egy Admin például:

\begin{itemize}
\item felhasználó jogának módosítása,
\item felhasználó törlése,
\item autók lekérdezése a többi oldalról.
\end{itemize}

Ehhez nagyon fontos a jogosultság vizsgálata, amit a @PreAuthorize annotációval valósítok meg amely SpEL (Spring Expression Language) használatával írhatók. Elle-
nőrzi a jogosultságot mielőtt be lépne a metódusba \cite{SpringSecurity}.

Az alábbi kódrész egy példa arra, hogy hogyan is használható az annotáció. Ennél a példánál a \textit{/set} végpontot csak az ADMIN jogú felhasználó érheti el.

\begin{java}
@GetMapping("/set")
@PreAuthorize("hasRole('ADMIN')")
public void getData() throws InterruptedException,
IOException {
   carService.saveCar();
}
\end{java}
\newpage

\subsection{JSON Web Token}
Mikor bejelentkezünk akkor a szerver generál egy JWT (JSON Web Token)\cite{JWTexample} tokent. Ez egy javasolt internetes szabvány az opcionálisan aláírt és/vagy opcionálisan titkosí-
tott adatok létrehozására, amelyek hasznos adattartalma bizonyos számú követelést deklaráló JSON-t tartalmaz. A tokeneket titkos kulccsal vagy nyilvános/titkos kulccsal írják alá.

A token tartalmazni fogja a felhasználó nevét. A token minden ki-és bejelentkezés során újragenerálódik. 
Az \ref{fig:JWT}. ábrán egy példa látható egy kódolt és dekódolt JWT-re.

\begin{figure}[h]
\centering
\includegraphics[scale=0.6]{images/jwt.io.png}
\caption{Kódolt és Dekódolt JWT \cite{JWTexample}}
\label{fig:JWT}
\end{figure}